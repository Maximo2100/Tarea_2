\onehalfspacing
\section{Análisis tiempo historia de la estructura}
Los resultados presentados en el programa ETABS se presentan en las siguientes secciones.

\subsection{Comparación de diagramas histeréticos con el gráfico teórico}

Se ha considerado que el resultado gráfico del diagrama histerético generado en el programa ETABS coincida con la curva bilineal teórica graficada en la figura \ref{i2_1} tanto para el caso del sistema total de asilamiento como para un asilador individual. Así, el resultado gráfico se muestra en la figura \ref{i4_1}.
\newpage
\begin{figure}[h!]
\centering
\begin{subfigure}[b]{0.48\linewidth}
\centering
\begin{tikzpicture}
    \begin{axis}[
        xtick={-40, -30, -20, -10, 0, 10, 20, 30, 40},
        ytick={-2800, -2000, -1200, -400, 400, 1200, 2000, 2800},
        width=7.8cm, height=6.5cm,
        ymin=-2800,ymax=2800,
        xmin=-40,xmax=40,
        minor tick num=4,
        grid style={line width=.1pt, draw=gray!10},
        major grid style={line width=.2pt,draw=gray!50},
        grid=both,
        %grid=major,
        tick label style={font=\tiny},
        clip=false,
        hide obscured x ticks=false,
        hide obscured y ticks=false,
        scaled ticks=false,
        axis lines=center,
        xlabel style={anchor=west,},
        xtick distance=1,
        xtick pos=bottom,
        ytick distance=1,
        ytick pos=left,
        %grid=major,
        xticklabel style={shift={(0,0 |- {axis description cs:0,-1})}},
        yticklabel style={shift={(0,0 -| {axis description cs:-0.50,0})}},
        ylabel={$F(tonf)$},
        xlabel={$x(cm)$},
        every axis x label/.style={
        at={(ticklabel* cs:1)},
        anchor=west,},
        every axis y label/.style={
            at={(ticklabel* cs:1)},
            anchor=south,},
        label style={font=\scriptsize{}},
        /pgf/number format/.cd,
        use comma,
        1000 sep={},
        legend style={at={(0.47,-0.26)}, anchor=north west, font=\scriptsize},
        ]
    \coordinate (sw) at (axis cs:-40,-2800);
    \coordinate (ne) at (axis cs:40,2800);
    \draw[line width=0.15pt] (sw) rectangle (ne);
    
    \addplot[mark=none, black, line width=1.25pt] coordinates {(0,589.13) (1.059, 654.60) (29.467,2410.50) (27.349, 1101.31) (0,-589.13) (-29.467,-2410.50) (-27.349, -1101.31) (0,589.13)};
    
    \addplot[forget plot, mark=none, black, line width=1.25pt] coordinates {(-1.059,-654.59) (1.059,654.59)};
    
    \addplot[forget plot, dashed, mark=none, black!75, line width=0.25pt] coordinates {(-29.467,-2410.50) (29.467,2410.50)};

    \addplot [blue,mark=none, line width = 1pt] table[col sep=semicolon] {HIST_ETABS/HISTE_SIST.csv};

    \legend{\scriptsize Curva teórica, \scriptsize Curva de ETABS}
    \end{axis}
\end{tikzpicture}
\caption{}
\label{i4_1.1}
\end{subfigure}
\hfill
\begin{subfigure}[b]{0.48\linewidth}
\centering
\begin{tikzpicture}
    \begin{axis}[
        xtick={-40, -30, -20, -10, 0, 10, 20, 30, 40},
        ytick={-40, -30, -20, -10, 0, 10, 20, 30, 40},
        width=7.8cm, height=6.5cm,
        ymin=-40,ymax=40,
        xmin=-40,xmax=40,
        minor tick num=4,
        grid style={line width=.1pt, draw=gray!10},
        major grid style={line width=.2pt,draw=gray!50},
        grid=both,
        %grid=major,
        tick label style={font=\tiny},
        clip=false,
        hide obscured x ticks=false,
        hide obscured y ticks=false,
        scaled ticks=false,
        axis lines=center,
        xlabel style={anchor=west,},
        xtick distance=1,
        xtick pos=bottom,
        ytick distance=1,
        ytick pos=left,
        xticklabel style={shift={(0,0 |- {axis description cs:0,-1})}},
        yticklabel style={shift={(0,0 -| {axis description cs:-0.507,0})}},
        ylabel={$F(tonf)$},
        xlabel={$x(cm)$},
        every axis x label/.style={
        at={(ticklabel* cs:1)},
        anchor=west,},
        every axis y label/.style={
            at={(ticklabel* cs:1)},
            anchor=south,},
        label style={font=\scriptsize{}},
        /pgf/number format/.cd,
        use comma,
        1000 sep={},
        legend style={at={(0.47,-0.26)}, anchor=north west, font=\scriptsize},
        ]
    \coordinate (sw) at (axis cs:-40,-40);
    \coordinate (ne) at (axis cs:40,40);
    \draw[line width=0.15pt] (sw) rectangle (ne);
    
    \addplot[mark=none, black!65, line width=1.25pt] coordinates {(0,6.54) (1.059, 7.27) (29.467,26.78) (27.349, 12.24) (0,-6.54) (-29.467,-26.78) (-27.349, -12.24) (0,6.54)};
    
    \addplot[forget plot, mark=none, black!65, line width=1.25pt] coordinates {(-1.059,-7.27) (1.059,7.27)};
    
    \addplot[forget plot, dashed, mark=none, black!65, line width=0.5pt] coordinates {(-29.467,-26.78) (29.467,26.78)};
    
    \addplot [blue!70,mark=none, line width = 1pt] table[col sep=semicolon] {HIST_ETABS/HISTE_DISP.csv};

    \legend{\scriptsize Curva teórica, \scriptsize Curva de ETABS}
    \end{axis}
\end{tikzpicture}
\caption{}
\label{i4_1.2}
\end{subfigure}
\caption[Comparación gráfica entre curvas histeréticas teóricas y curvas generadas por ETABS]{Comparación gráfica entre curvas histeréticas teóricas y curvas generadas por ETABS:
    \subref{i4_1.1} Sistema de aislamiento;
    \subref{i4_1.2} Aislamiento individual
}
\label{i4_1}
\end{figure}


Se ha analizado el aislador $K_{50}$ y se ha considerado la gráfica generada por el sismo de Lima 1974-EW al 100\%. Este se encuentra ubicado en la parte central de la base de aislamiento. Asimismo, se observa que las curvas histeréticas forman trazos similares a la curva bilineal hallada de forma teórica en la sección 2.1.6.

\subsection{Desplazamientos máximos del dispositivo (aislador)}

Se ha considerado al aislador con etiqueta $K_{90}$ y se ha determinado el máximo desplazamiento de cada caso de sismo. Para obtener el máximo desplazamiento se ha tomado en cuenta la siguiente expresión:
\begin{equation}
    D_{a-max} = \sqrt{D_x^2 + D_y^2} \label{eq4_1}
\end{equation}
Donde:
\begin{itemize}
    \item{\makebox[1.65cm][l]{$D_{ais-max}$}}: desplazamiento del aislador proveniente de la raiz de la suma de cuadrados
    \item{\makebox[1.65cm][l]{$D_x$}}: desplazamiento del aislador en la dirección $\vec{x}$
    \item{\makebox[1.65cm][l]{$D_x$}}: desplazamiento del aislador en la dirección $\vec{y}$
\end{itemize}

Cada registro sísmico genera puntos y de la ecuación anterior se genera un valor; del total de puntos se elige el máximo valor.\\

Los gráficos generados y su correspondiente desplazamiento máximo en cada registro sísmico se muestra a continuación:

\newpage
\begin{figure}[H]
    \centering
\begin{subfigure}[b]{0.99\linewidth}
\centering
\begin{tikzpicture}
    \begin{axis}[
        xtick={0, 10, 20, 30, 40, 50, 60, 70},
        ytick={0, 2, 4, 6, 8, 10, 12, 14, 16, 18, 20, 22},
        width=0.99\linewidth, height=5.7cm,
        xlabel={$t \; (seg)$},
        ylabel={$D \; (cm)$},
        label style={font=\normalsize},
        tick label style={font=\tiny},
        x label style={at={(axis description cs: 0.5,0)},anchor=north},
        y label style={at={(axis description cs:0,.5)},anchor=north},
        grid style={line width=.1pt, draw=gray!10},
        major grid style={line width=.2pt,draw=gray!50},
        grid = both,
        minor tick num=4,
        xmin = 0.0,
        xmax = 75,
        ymin = 0,
        ymax = 22,
        clip=false,]
    \addplot [black,mark=none, line width = 0.75pt] table[col sep=semicolon] {DESP_AISL/AR_EW.csv};
    
    \draw[red, line width=0.25pt] (axis cs:39.76,20.33) circle [radius=4pt];
    \draw[blue!70, smooth, line width=0.25pt] (axis cs:39.76,20.33) to[out=-30, in=180] (axis cs:50,15) node[right,font=\small] {$(t=39.76,D_{max}=20.33)$};
    \legend{\small $D_{max}$ sismo AR-2001-EW};
    \end{axis}
  \end{tikzpicture}
  \caption{}
  \label{i4_2.1}
\end{subfigure}
\hfill
\begin{subfigure}[b]{0.99\linewidth}
\centering
\begin{tikzpicture}
    \begin{axis}[
        xtick={0, 5, 10, 15, 20, 25, 30, 35, 40, 45},
        ytick={0, 2, 4, 6, 8, 10, 12, 14, 16, 18, 20, 22, 24, 26},
        width=0.99\linewidth, height=5.7cm,
        xlabel={$t \; (seg)$},
        ylabel={$D \; (cm)$},
        label style={font=\normalsize},
        tick label style={font=\tiny},
        x label style={at={(axis description cs: 0.5,0)},anchor=north},
        y label style={at={(axis description cs:0,.5)},anchor=north},
        grid style={line width=.1pt, draw=gray!10},
        major grid style={line width=.2pt,draw=gray!50},
        grid = both,
        minor tick num=4,
        xmin = 0.0,
        xmax = 45,
        ymin = 0,
        ymax = 26,
        clip=false,]
    \addplot [black,mark=none, line width = 0.75pt] table[col sep=semicolon] {DESP_AISL/LIMA66_EW.csv};
    
    \draw[red, line width=0.25pt] (axis cs:22.12,25.342) circle [radius=4pt];
    \draw[blue!70, smooth, line width=0.25pt] (axis cs:22.12,25.342) to[out=-40, in=180] (axis cs:30,15) node[right,font=\small] {$(t=22.12,D_{max}=25.34)$};
    \legend{\small $D_{max}$ sismo LIMA-1966-EW};
    \end{axis}
  \end{tikzpicture}
  \caption{}
  \label{i4_2.2}
\end{subfigure}
\hfill
\begin{subfigure}[b]{0.99\linewidth}
\centering
\begin{tikzpicture}
    \begin{axis}[
        xtick={0, 10, 20, 30, 40, 50, 60, 70, 80, 90},
        ytick={0, 2, 4, 6, 8, 10, 12, 14, 16, 18, 20, 22, 24},
        width=0.99\linewidth, height=5.7cm,
        xlabel={$t \; (seg)$},
        ylabel={$D \; (cm)$},
        label style={font=\normalsize},
        tick label style={font=\tiny},
        x label style={at={(axis description cs: 0.5,0)},anchor=north},
        y label style={at={(axis description cs:0,.5)},anchor=north},
        grid style={line width=.1pt, draw=gray!10},
        major grid style={line width=.2pt,draw=gray!50},
        grid = both,
        minor tick num=4,
        xmin = 0.0,
        xmax = 90,
        ymin = 0,
        ymax = 24,
        clip=false,]
    \addplot [black,mark=none, line width = 0.75pt] table[col sep=semicolon] {DESP_AISL/LIMA74_EW.csv};
    
    \draw[red, line width=0.25pt] (axis cs:13.14,23.51) circle [radius=4pt];
    \draw[blue!70, smooth, line width=0.25pt] (axis cs:13.14,23.51) to[out=-10, in=180] (axis cs:30,15) node[right,font=\small] {$(t=13.14,D_{max}=23.51)$};
    \legend{\small $D_{max}$ sismo LIMA-1974-EW};
    \end{axis}
  \end{tikzpicture}
  \caption{}
  \label{i4_2.3}
\end{subfigure}
\hfill
\begin{subfigure}[b]{0.99\linewidth}
\centering
\begin{tikzpicture}
    \begin{axis}[
        xtick={0, 10, 20, 30, 40, 50, 60, 70, 80, 90, 100, 110, 120},
        ytick={0, 2, 4, 6, 8, 10, 12, 14, 16, 18, 20, 22},
        width=0.99\linewidth, height=5.7cm,
        xlabel={$t \; (seg)$},
        ylabel={$D \; (cm)$},
        label style={font=\normalsize},
        tick label style={font=\tiny},
        x label style={at={(axis description cs: 0.5,0)},anchor=north},
        y label style={at={(axis description cs:0,.5)},anchor=north},
        grid style={line width=.1pt, draw=gray!10},
        major grid style={line width=.2pt,draw=gray!50},
        grid = both,
        minor tick num=4,
        xmin = 0.0,
        xmax = 120,
        ymin = 0,
        ymax = 22,
        clip=false,]
    \addplot [black,mark=none, line width = 0.75pt] table[col sep=semicolon] {DESP_AISL/MAULE_EW.csv};
    
    \draw[red, line width=0.25pt] (axis cs:40.70,21.43) circle [radius=4pt];
    \draw[blue!70, smooth, line width=0.25pt] (axis cs:40.70,21.43) to[out=-10, in=180] (axis cs:60,15) node[right,font=\small] {$(t=40.70,D_{max}=21.43)$};
    \legend{\small $D_{max}$ sismo MAULE-2010-EW};
    \end{axis}
  \end{tikzpicture}
  \caption{}
  \label{i4_2.4}
\end{subfigure}
\end{figure}

\newpage

\begin{figure}[H]
    \centering
    \ContinuedFloat
\begin{subfigure}[b]{0.99\linewidth}
\centering
\begin{tikzpicture}
    \begin{axis}[
        xtick={0, 10, 20, 30, 40, 50, 60, 70, 80, 90, 100, 110, 120},
        ytick={0, 2, 4, 6, 8, 10, 12, 14, 16, 18, 20, 22, 24},
        width=0.99\linewidth, height=5.8cm,
        xlabel={$t \; (seg)$},
        ylabel={$D \; (cm)$},
        label style={font=\normalsize},
        tick label style={font=\tiny},
        x label style={at={(axis description cs: 0.5,0)},anchor=north},
        y label style={at={(axis description cs:0,.5)},anchor=north},
        grid style={line width=.1pt, draw=gray!10},
        major grid style={line width=.2pt,draw=gray!50},
        grid = both,
        minor tick num=4,
        xmin = 0.0,
        xmax = 120,
        ymin = 0,
        ymax = 24,
        clip=false,]
    \addplot [black,mark=none, line width = 0.75pt] table[col sep=semicolon] {DESP_AISL/PISCO_EW.csv};
    
    \draw[red, line width=0.25pt] (axis cs:21.04,22.90) circle [radius=4pt];
    \draw[blue!70, smooth, line width=0.25pt] (axis cs:21.04,22.90) to[out=-10, in=180] (axis cs:40,17) node[right,font=\small] {$(t=21.04,D_{max}=22.90)$};
    \legend{\small $D_{max}$ sismo PISCO-2007-EW};
    \end{axis}
  \end{tikzpicture}
  \caption{}
  \label{i4_2.5}
\end{subfigure}
\hfill
\begin{subfigure}[b]{0.99\linewidth}
\centering
\begin{tikzpicture}
    \begin{axis}[
        xtick={0, 10, 20, 30, 40, 50, 60, 70},
        ytick={0, 2, 4, 6, 8, 10, 12, 14, 16, 18, 20, 22, 24, 26, 28, 30},
        width=0.99\linewidth, height=5.8cm,
        xlabel={$t \; (seg)$},
        ylabel={$D \; (cm)$},
        label style={font=\normalsize},
        tick label style={font=\tiny},
        x label style={at={(axis description cs: 0.5,0)},anchor=north},
        y label style={at={(axis description cs:0,.5)},anchor=north},
        grid style={line width=.1pt, draw=gray!10},
        major grid style={line width=.2pt,draw=gray!50},
        grid = both,
        minor tick num=4,
        xmin = 0.0,
        xmax = 70,
        ymin = 0,
        ymax = 30,
        clip=false,]
    \addplot [black,mark=none, line width = 0.75pt] table[col sep=semicolon] {DESP_AISL/TARAPACA_EW.csv};
    
    \draw[red, line width=0.25pt] (axis cs:14.78,29.33) circle [radius=4pt];
    \draw[blue!70, smooth, line width=0.25pt] (axis cs:14.78,29.33) to[out=-10, in=180] (axis cs:25,15) node[right,font=\small] {$(t=14.78,D_{max}=29.33)$};
    \legend{\small $D_{max}$ sismo TARAPACÁ-2005-EW};
    \end{axis}
  \end{tikzpicture}
  \caption{}
  \label{i4_2.6}
\end{subfigure}
\hfill
\begin{subfigure}[b]{0.99\linewidth}
\centering
\begin{tikzpicture}
    \begin{axis}[
        xtick={0, 10, 20, 30, 40, 50, 60},
        ytick={0, 2, 4, 6, 8, 10, 12, 14, 16, 18, 20, 22},
        width=0.99\linewidth, height=5.8cm,
        xlabel={$t \; (seg)$},
        ylabel={$D \; (cm)$},
        label style={font=\normalsize},
        tick label style={font=\tiny},
        x label style={at={(axis description cs: 0.5,0)},anchor=north},
        y label style={at={(axis description cs:0,.5)},anchor=north},
        grid style={line width=.1pt, draw=gray!10},
        major grid style={line width=.2pt,draw=gray!50},
        grid = both,
        minor tick num=4,
        xmin = 0.0,
        xmax = 60,
        ymin = 0,
        ymax = 22,
        clip=false,]
    \addplot [black,mark=none, line width = 0.75pt] table[col sep=semicolon] {DESP_AISL/TOCOPILLA_EW.csv};
    
    \draw[red, line width=0.25pt] (axis cs:17.64,20.55) circle [radius=4pt];
    \draw[blue!70, smooth, line width=0.25pt] (axis cs:17.64,20.55) to[out=-10, in=180] (axis cs:25,15) node[right,font=\small] {$(t=17.64,D_{max}=20.55)$};
    \legend{\small $D_{max}$ sismo TOCOPILLA-2007-EW};
    \end{axis}
  \end{tikzpicture}
  \caption{}
  \label{i4_2.7}
\end{subfigure}
    \caption[Desplazamientos máximos del aislador $K_{90}$ a lo largo del tiempo genererados por el registro sísmico en dirección Este-Oeste (EW)]{Desplazamientos máximos del aislador $K_{90}$ a lo largo del tiempo genererados por el registro sísmico en dirección Este-Oeste (EW):
    \subref{i4_2.1} AR-2001;
    \subref{i4_2.2} LIMA-1966;
    \subref{i4_2.3} LIMA-1974;
    \subref{i4_2.4} MAULE-2010;
    \subref{i4_2.5} PISCO-2007;
    \subref{i4_2.6} TARAPACÁ-2005;
    \subref{i4_2.7} TOCOPILLA-2007
    }
    \label{i4_2}
\end{figure}

%\newline
Asimismo, la tabla que resume los valores máximos e indica el promedio de estos 7 registros se muestra a continuación:

\begin{align*}
    D_{prom} &= \dfrac{|D_1| + |D_2|}{2} \\
    F_{prom} &= \dfrac{|F_1| + |F_2|}{2} \\
    K_m &= \dfrac{F_{prom}}{D_{prom}} \\
    A &= 4Q \cdot (D_{prom} - D_y) \\
    \beta_m &= \dfrac{A}{2\pi \cdot K_m \cdot D_{prom}^2}
\end{align*}

PROBANDO ... PROBANDO ... PROBANDO 222222222222
SSSSSS

\begin{comment}

En cada curva de desplazamientos se ha determinado el máximo valor. Así para cada registro sísmico se tiene la siguiente tabla de valores:

\begin{longtblr}[
    caption = {Lista de acelerogramas ingresados al programa ETABS},
    label = {t4_1},
]{
    colspec = {r X[l,wd=4cm] c r},
    row{1} = {font = \bfseries},
    hline{1,10} = {1pt},
    hline{2,9} = {0.25pt},
    row{3,5,7} = {gray!8},
}
    n\textdegree & Nombre & Orientación & Desp. $\bm{(cm)}$ \\

    1 & SM-AR-2001 & EW & $37.827$ \\

    2 & SM-LIMA-1966 & EW & $55.421$ \\

    3 & SM-LIMA-1974 & EW & $49.355$ \\

    4 & SM-MAULE-2010 & EW & $37.927$ \\

    5 & SM-PISCO-2007 & EW & $40.337$ \\

    6 & SM-TARAPACA-2005& EW & $50.764$ \\

    7 & SM-TOCOPILLA-2007 & EW & $38.805$ \\
    
    {} & {} & Promedio & $44.348$
\end{longtblr}


\begin{SISMO}{SISMO LIMA 1974}
  $f\in C^{1}(D,\mathbb{R})$. Dann gibt es auf jeder Strecke
  $[x_0,x]\subset D$ einen Punkt $\xi\in[x_0,x]$, so dass gilt
    \begin{equation*}
    f(x)-f(x_0) = \operatorname{grad} f(\xi)^{\top}(x-x_0)
    \end{equation*}
\end{SISMO}
\end{comment}